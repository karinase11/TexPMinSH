\chapter*{Kurzbeschreibung}\vspace{17mm}
Nachfolgende Bachelorarbeit untersucht die Eignung von Process Mining Algorithmen für den Einsatz in Smart Living Szenarien. Hierfür werden die gängigsten Process Mining Algorithmen durch die Durchführung einer Testreihe miteinander verglichen und daraufhin untersucht, ob sie sich zur Vorhersage von repetitivem menschlichen Verhalten eignen, welches durch vernetzte Geräte im Haushalt aufgezeichnet wurde. Als Proof of Concept wird eine Vorhersagefunktion in einer Android Anwendung integriert, die einem Nutzer Vorschläge in der sogenannten If This Then That Form, basierend auf den Ergebnissen des Mining Algorithmus, macht und in der Lage ist diese Regel selbstständig im Smart Home System zu hinterlegen. Für die automatisierte Regelerkennung wird eine Skript angelegt, welches die Durchführung der Analyse in einem lokalen Netzwerk mit geringem Rechenressourcenverbrauch erlaubt. 

Ziel einer solchen Anwendung ist die Erleichterung des Umgangs mit der wachsenden Anzahl vernetzter Geräte im privaten Raum, indem dem Verbraucher Konfigurationsschritte durch intelligente Analyse abgenommen werden, ohne dass dabei private Daten an dritte Instanzen gelangen. 

Die Ergebnisse der Testreihe deuten darauf hin, dass das Heuristic Miner Verfahren geeignet ist, gesuchte Regeln so zu modellieren, dass sie in eine Anwendung überführt werden können, die dann das Potential hat den Komfort des Bewohners eines Smart Home zu erhöhen.\todo{ausbessern}