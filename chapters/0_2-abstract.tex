\chapter*{Kurzbeschreibung}\vspace{17mm}
Die vorliegende Bachelorarbeit hat die Eignung von Process Mining Algorithmen für den Einsatz in Smart Living Szenarien zum Thema. Hierfür werden die gängigsten Process Mining Algorithmen in einer Testreihe miteinander verglichen und daraufhin untersucht, ob sie sich zur Vorhersage von repetitivem menschlichen Verhalten eignen, das durch vernetzte Geräte im Haushalt aufgezeichnet wurde. Als Proof of Concept wird eine Vorhersagefunktion in eine Android-Anwendung integriert, die einem Anwender Vorschläge in der sogenannten If This Then That Form, basierend auf den Ergebnissen eines Process Mining Algorithmus, unterbreitet und in der Lage ist, diese Regel selbstständig im Smart Home System zu hinterlegen. Für die automatisierte Regelerkennung wird ein Skript umgesetzt, das die Durchführung der Analyse in einem lokalen Netzwerk und mit geringem Rechenressourcenverbrauch erlaubt.

Ziel einer solchen Anwendung ist die Erleichterung des Umgangs mit der wachsenden Anzahl vernetzter Geräte im privaten Raum, indem der Verbraucherin oder dem Verbraucher Konfigurationsschritte durch intelligente Analyse abgenommen werden, ohne dass dabei private Daten an Dritte gelangen.

Die Ergebnisse der Testreihe deuten darauf hin, dass das Heuristic Miner Verfahren geeignet ist, gesuchte Prozesse derart zu modellieren, dass sie in eine Regel überführt werden können, die dann das Potential hat, den Komfort der Bewohnerin oder des Bewohners eines Smart Homes zu erhöhen.

\newpage
\chapter*{Abstract}\vspace{17mm}
The aim of this thesis is to examine the suitability of process mining algorithms for use in smart living scenarios. For this purpose, the most common process mining algorithms are compared with each other in a test series and examined as to whether they are suitable for predicting repetitive human behavior, which is  recorded by IoT devices in the household. As a proof of concept, a prediction function is integrated into an Android application, which makes suggestions to the user in the so-called If This Then That Form, based on the results of a process mining algorithm. 
For automated rule recognition, a script is implemented that allows the process mining analysis to be performed in a local network, requiering only low computing resources. 

The aim of such an application is to facilitate the handling of the growing number of networked devices in private spaces. This is achieved by removing configuration steps from the consumer through intelligent analysis, without private data being passed on to third parties. 

The results of the test series indicate that the Heuristics Miner method is suitable for modelling short patterns of human behavior in such a way that they can be converted into. This, in turn, has the potential to increase the comfort of the occupant of a smart home.
