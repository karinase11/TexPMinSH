\chapter{Einleitung}\label{chap:introduction}
Mit einem wachsenden Angebot an Smart Home Geräten auf dem Weltmarkt nimmt auch die Anzahl der Plattformen stetig zu, die Lösungen für die Steuerung von vernetzten Geräten anbieten. Mit dieser wachsenden Vielfalt erhöht sich auch der Bedarf an benutzerfreundlichen Schnittstellen für die Kommunikation mit dem intelligenten Eigenheim. Dass das Smart Home zunehmend an Relevanz gewinnt und zukünftig allgegenwärtig sein wird belegen unter anderem die Zahlen des Marktforschungsunternehmens IDC. Das Wachstem des weltweiten Smart Home Markts im Jahr 2019 wird auf voraussichtlich 26,9 \% und 832,7 Millionen verkaufte Geräte geschätzt \cite{IDC}.
 
Einer breiten Adaption von intelligenter Haustechnik stehen gegnwärtig noch einige Vorbehalte der Nutzer entgegen. Hemmnisse für die Akzeptanz von Smart Home Systemen betreffen vordergründig Bedenken hinsichtlich des Datenschutzes und der Datensicherheit, so die Herausgeber einer Studie des Marktforschungsinstituts Dr. Grieger \& Cie \cite{griegercie}. Das IDC hält auch in Anbetracht der Bedenken seitens der Verbraucher an einer positiven Marktprognose fest, was durch Aussagen der Befragten begründet wird, wonach der Zugewinn an Komfort durch Smart Home Geräte die Sorgen bezüglich der Datensicherheit häufig überwiege.
 
Um den Ansprüchen nach mehr Komfort gerecht zu werden, müssen intelligente Systeme nicht nur auf die vom Nutzer geäußerte Wünsche reagieren, sondern dessen Bedürfnisse antizipieren, so die Maxime von marktführenden Herstellern wie Amazon oder Google \cite{IoTGoogle}. Um dies zu ermöglichen werden IoT Geräte durch Verfahren aus den Forschungsfeldern der künstlichen Intelligenz und der Big Data Analyse gestützt, die eine Auswertung der gesammelten Daten über die Interaktion des Nutzers mit IoT Geräten erlauben und in der Lage sind Prognosen zur Nutzung der Geräte zu treffen. 
\newpage
Ein solches Verfahren, das dem Gebiet der Datenanalyse zuzuordnen ist, und seinen Ursprung unter Anderem in der Auswertung von Geräten in der maschinellen Produktion hat, ist das Process Mining. Process Mining ist ein Forschungsgebiet, das sowohl mit der Business Intelligence als auch mit Data Science Schnittmengen bildet \cite{PMinAction}. 

Es unterscheidet sich von konventionellem Data Mining insofern, als es sich vordergründig mit der Analyse von in der reellen Welt ablaufenden Prozessen beschäftigt. Es befasst sich nicht mit der Klassifizierung von Momentaufnahmen oder mit der Beantwortung von vergleichbaren instanzbezogenen Fragestellungen. Viele Technologien zur automatisierten Datenanalyse, wie etwa die Bilderkennung oder die linguistische Datenverarbeitung, beschäftigen sich mit Problemen, die unabhängig von einem zeitlichen Kontext zu betrachten sind. Anders verhält es sich beim Process Mining, dessen Hauptaugenmerk auf der sogenannten \textit{process awareness}, also dem Prozessbewusstsein, liegt.
\section{Fragestellung}
Ein Smart Home beschreibt ein System, das den Komfort im Eigenheim durch technische Lösungen erhöht, indem ein Netzwerk aus untereinander oder mit dem Nutzer interagierenden elektronischen Geräten im Gebäude integriert wird. Diese Geräte verfügen, neben ihrer inhärenten Funktionalität, auch über die Fähigkeit einen Logeintrag aus jeder ihrer Interaktionen oder Sensoraufnahmen zu generieren. Der Umstand, dass Process Mining Algorithmen zu erst Protokolldateien derjenigen Umgebungen benötigen, die sie analysieren sollen, legt die Kombination von Process Mining und dem System Smart Home nahe.
 
Bisher ist diese technologische Schnittstelle noch nicht tiefgreifend erforscht. Es sind nur vereinzelt Studien veröffentlicht worden, die die genannten Technologien miteinander kombinieren. Die vorliegende Arbeit soll klären, ob bestehende Process Mining Algorithmen in einem Smart Home effektiv eingesetzt werden können, um menschliches Verhalten korrekt zu abstrahieren. Ziel ist es dabei die Analyse kleiner Datensätze mit existierenden Softwarelösungen so zu gestalten, dass sie dem Benutzerkomfort im Umgang mit IoT Geräten zuträglich sind. Zweck der Architektur ist es die Auswertung gleichzeitig so ressourcenschonend aufzubauen, dass private Daten des Nutzers das lokale Netz nicht verlassen müssen.

\section{Zielsetzung}
Im Detail soll erarbeitet werden, ob Process Mining dazu eingesetzt werden kann, um menschliches Verhalten auf kurze, wiederkehrende Verhaltensmuster hin zu untersuchen und welches der gängigen Process Mining Verfahren für diesen Einsatzzweck besonders geeignet erscheint. Für eine Vorauswahl werden zunächst Studien betrachtet, die Process Mining bereits im Bereich des Smart Living eingesetzt haben. Anschließend werden diese systematisch in einer Testreihe miteinander verglichen, um festzustellen welche Leistungsfähigkeit von den etablierten Process Mining Algorithmen in dem darstellbaren Smart Living Szenario erwartet werden kann. Wichtige Kriterien sind dabei die Fehleranfälligkeit, die Rate vollständig akkurat erkannter Muster, sowie der Umgang mit Rauschdaten und parallel verlaufenden Prozessen.

Als Simulationsumgebung für den Einsatz des Programms soll ein Modell eines Smart Home dienen, das im Rahmen einer Bachelorarbeit am IT Institut für Organisation und Management der Fachhochschule Aachen entstanden ist und über die quelloffene Software openHab\footnote{https://www.openhab.org/docs/} gesteuert wird. Nach Auswertung der Protokolldateien nach unterschiedlichen Gesichtspunkten sollen die Erkenntnisse genutzt werden, um einen sogenannten Proof of Concept\footnote{Im Projektmanagement ist ein Proof of Concept, auch als Proof of Principle bezeichnet, ein Meilenstein, an dem die prinzipielle Durchführbarkeit eines Vorhabens belegt ist. In der Regel ist mit dem Proof of Concept meist die Entwicklung eines Prototyps verbunden, der die benötigte Kernfunktionalität aufweist.\cite{poc}} zu erstellen. Dieser wird auf einem einfachen Einplatinenrechner realisiert, der über eine Android App mit dem Anwender interagiert.

Ein Vorteil, den Process Mining dem Smart Home Anwender bieten kann, liegt in der automatisierten Erkennung von Prozessmustern im Umgang mit IoT Geräten. Bisher bieten eine Reihe von Unternehmen Plattformen an, die es dem Anwender lediglich durch manuelle Eingabe von Abläufen ermöglichen, Prozesse im Eigenheim zu automatisieren. Häufig findet dafür das sogenannte If This Then That Muster Anwendung, wie es beispielsweise in Webthings von Mozilla\footnote{IoT.mozilla.org},  oder Stringify\footnote{stringify.com} eingesetzt wird. 
\newpage
If This Then That beschreibt das Schema einer Regel, die aus Bedingungen bestehet, die jeweils vorkonfigurierte Aktionen auslösen. Diese müssen vom Anwender selbstständig als wiederkehrend erkannt und über eine Eingabemaske konfiguriert werden. Wird die Fähigkeit von Process Mining genutzt, Muster in aufgezeichneten Protokolldateien zu erkennen, bietet sich diese Methode an, um dem Anwender Konfigurationen in seinem Smart Home automatisch vorzuschlagen. Er kann dann die vorgeschlagene Regel bestätigen oder ablehnen. Zum einen hätte diese Funktion den Vorteil, dass es weniger Konfigurationsaufwand seitens des Anwenders bedarf, um einen neuen automatisierten Ablauf in seinem Smart Home zu integrieren, zum anderen ist es so möglich, dass ein Algorithmus Muster erkennt, die der Aufmerksamkeit des Anwenders entgangen wären, aber durch eine Implementierung den Komfort im Smart Home erhöhen.

%Die Ergebnisse aus der Auswertung des Process Mining auf den Smart Home Daten sollen nicht allein Muster im Tagesablauf des Anwenders erkennen, sondern auch Zusammenhänge zwischen gemessenen Sensorwerten wie etwa aus  Temperatur- oder Helligkeitssensoren und den Nutzeraktionen feststellen. Dies kann zahlreiche weitere Konfigurationsschritte seitens des Anwenders erübrigen und hat das Potential, es dem Anwender zu erlauben die Möglichkeiten des Smart Homes weiter auszuschöpfen, als es bei rein manueller Eingabe möglich wäre, da nicht jedes Muster vom Anwender bewusst wahrgenommen wird, welches Teil seiner Routine im Smart Home geworden ist.

%\section{Template Structure}
%\begin{itemize}
%    \item To compile the document either run the makefile or run your compiler on the file %`thesis\_main.tex'. The included makefile requires latexmk which automatically runs %bibtex and recompiles your thesis as often as needed. Also it automatically places all %output files (aux, bbl, ...) in the folder `out'. As the pdf also goes in there, the %makefile copies the pdf file to the parent folder. There is also a makefile in the %chapters folder, to ensure you can also compile from this directory.
%
%    \item The file `setup.tex' includes the packages and defines commands. For more details %see \secref{sec:setup}.
%
%    \item Each chapter goes into a separate document, the files can be found in the folder %chapters.
%
%    \item The bib folder contains the .bib files, I'd suggest to create multiple bib files %for different topics. If you add some or rename the existing ones, don't forget to also %change this in thesis\_main.tex. You can then cite as usual~\cite{kingma2014adam, %bromley1993siamesesignature,muja2009flann}.
%
%    \item The template is written in a way that eases the switch from scrbook to book %class. So if you're not a fan of KOMA you can just replace the documentclass in the %main file. The only thing that needs to be changed in setup.tex is the caption styling, %see the comments there.
%\end{itemize}
%
%
%\section{setup.tex}\label{sec:setup}
%Edit setup.tex according to your needs. The file contains two sections, one for package %includes, and one for defining commands. At the end of the includes and commands there is a %section that can safely be removed if you don't need algorithms or tikz. Also don't forget %to adapt the pdf hypersetup!!\\
%setup.tex defines:
%\begin{itemize}
%    \item some new commands for remembering to do stuff:
%    \begin{itemize}
%        \item \verb|\todo{Do this!}|: \todo{Do this!}
%        \item \verb|\extend{Write more when new results are out!}|:\\ \extend{Write more %when new results are out!}
%        \item \verb|\draft{Hacky text!}|: \draft{Hacky text!}
%    \end{itemize}
%
%    \item some commands for referencing, `in \verb|\chapref{chap:introduction}|' produces %'in \chapref{chap:introduction}'
%    \begin{itemize}
%        \item \verb|\chapref{}|
%        \item \verb|\secref{sec:XY}|
%        \item \verb|\eqref{}|
%        \item \verb|\figref{}|
%        \item \verb|\tabref{}|
%    \end{itemize}
%
%    \item the colors of the Uni's corporate design, accessible with\\ \verb|{\color{UniX} %Colored Text}|
%    \begin{itemize}
%        \item {\color{UniBlue}UniBlue}
%        \item {\color{UniRed}UniRed}
%        \item {\color{UniGrey}UniGrey}
%    \end{itemize}
%
%    \item a command for naming matrices \verb|\mat{G}|, $\mat{G}$, and naming vectors %\verb|\vec{a}|, $\vec{a}$. This overwrites the default behavior of having an arrow over %vectors, sticking to the naming conventions  normal font for scalars, bold-lowercase %for vectors, and bold-uppercase for matrices.
%
%    \item named equations:
%        \begin{verbatim}
%\begin{align}
%    d(a,b) &= d(b,a)\\ \eqname{symmetry}
%\end{align}
%        \end{verbatim}
%        \begin{align}
%            d(a,b) &= d(b,a)\\ \eqname{symmetry}
%        \end{align}
%\end{itemize}
%
%\section{Advice}\label{sec:advice}
%This section gives some advice how to write a thesis ranging from writing style to %formatting. To be sure, ask your advisor about his/her preferences.\\
%For a more complete list we recommend to read Donald Knuth's paper on mathematical writing. %(At least the first paragraph). %\url{http://jmlr.csail.mit.edu/reviewing-papers/knuth_mathematical_writing.pdf}
%
%
%        \item Usually  in a thesis you don't write `In [24] the data is..'. You have more %space than a paper has, so write `AuthorXY et al. prepare the data... [24]'. Also %pay attention to the placement: The citation is at the end of the sentence before %the full stop with a no-break space. \verb|... last word~\cite{XY}.|
%
%
%
%
%        \item Use \verb|``''| for citing, not \verb|""|.
%
%    \end{itemize}
%