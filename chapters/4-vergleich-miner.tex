\chapter{Untersuchung geeigneter Process Miner}\label{chap:approach}
Um systematisch einen geeigneten Algorithmus zu finden, der automatisiert Regeln aus den Protokolldateien eines Smart Home extrahiert, soll eine begrenze Anzahl an Process Mining Verfahren in Testdurchläufen miteinander verglichen werden. Der Testaufbau ist derart gestaltet, dass dieser sich mit einem preisgünstigen Einplatinencomputer, wie etwa des Herstellers Raspberry PI, in einem realen Smart Home realisieren ließe. Zunächst soll eine Software für die Ausführung des Process Mining Verfahrens gefunden werden.

\section{Auswahl der Softwarelösung}
Essentiell für die Analyse ist ein Werkzeug, mit dem sich die Rohdaten über einen einheitlichen, vom verwendeten Algorithmus unabhängigen, Workflow verarbeiten lassen. Um eine geeignete Softwarelösung für die Durchführung der Process Mining Auswertung zu bestimmen, werden eine Reihe von Publikationen herangezogen, die  Process Mining Software, unter für den Anwendungsfall relevanten Gesichtspunkten, verglichen haben. Sie sollen dazu beitragen eine Vorauswahl unter den gegenwärtig verfügbaren Process Mining Lösungen zu treffen, aus der dann über einen Kriterienkatalog eine Software für den Einsatz in der Auswertung gewählt wird. 
Zunächst werden die Ergebnisse einer Abschlussarbeit der Universität Gent mit dem Titel „Process Mining in Practice: Comparative Study of Process Mining Software“ \cite{verstraete} betrachtet. Hier wurde anhand von Umfragen, unter Forschern und Anwendern in der freien Wirtschaft, untersucht, welche der neun genanneten Process Mining Lösungen den Befragten persönlich bereits bekannt war, ob und wie häufig sie von ihnen eingesetzt wurde und ob das Programm den Befragten nutzerfreundlich erschien.

Hier zeichnen sich ProM und Disco als Führend unter den vorgestellten Softwarelösungen ab. ProM wird der Umfrage nach bei von Befragten aus der Forschung besonders häufig einesetzt, während Anwender aus der Industrie meist auf kommerzielle Lösungen zurückgreifen, wie in Tabelle \ref{comparison} zu sehen ist.

\begin{table}[!h]
\centering
\resizebox{0.57\textwidth}{!}{%
\begin{tabular}{l  c c c }
\hline
Einsatzhäufigkeit & \textbf{Disco }& \textbf{ProM} & \textbf{Celonis} \\  
\hline\hline
\textit{häufig genutzt} & 26 &  13 & 2  \\
\textit{gelegentlich genutzt} & 13 &  17 & 0  \\
\textit{einmal genutzt} & 6 & 12 & 0 \\
\textit{bekannt} & 9  & 11 & 16  \\
\textit{unbekannt} & 1 & 2 & 34  \\
\hline
\end{tabular}%
}
\caption{Angaben von Anwendern zu ihrem Umgang mit Process Mining Software (Quelle: Auszug aus Tabelle 5: Fragenkatalog 5, Verstraete, Comparative Study of Process Mining Software, S. 29 \cite{compPM} }
\label{comparison}
\end{table}

\normalsize
Jedoch ist ProM, aufgrund der Tatsache, dass es sich um ein Open Source Projekt handelt, das mit Abstand am umfangreichsten ausgestattete Werkzeug,  hinsichtlich seiner Konfigurationmöglichkeiten und zahlreichen Erweiterungen. 
Im Laufe seiner Entwicklung wurden zahlreiche Plugins, die verschiedenene Process Mining Algorithmen implementieren, an das Projekt angebunden und die Auswahl an Ausgabeformaten stetig erweitert, so dass das Funktionsspektrum von ProM dem von Disco oder Celonis deutlich überlegen ist, so die Autoren. 

Eine ausführliche Analyse von Process Mining Software unter verschiedenen Gesichtspunkten ist in dem Buch "Advances in Intelligent Process-Aware Information Systems - Concepts, Methods, and Technologies" von J. Schobel und M.  Reichert, erschienen 2017 im Springer Verlag, enthalten\cite{Schobel2017}. Es werden auch, neben weiteren Anbietern, wieder Celonis, Disco und ProM in den Vergleich miteinbezogen, allerdings fassen die Autoren Disco und ProM als Einheit auf, da sie von der selben Organisation entwickelt wurden. Auch in dieser Auswertung sticht ProM durch seine Bandbreite an Erweiterungen hervor, wird von den Autoren aber für die mangelnde Echtzeitüberwachung abgemahnt, die Celonis anbietet.

ProM, Disco sowie Celonis werden außerdem, mit Abstand zu den anderen verglichenen Lösungen, positiv für die Umsetzung der Process Discovery Funktion bewertet, was die Entscheidung sie in die nähe Auswahl zu nehmen bestärkt. Eine graphische Übersicht über die Ergebnisse der Auswertung der Autoren ist in Abbildung \ref{fig:toolEval} zu sehen.
\begin{figure}[!ht]
    \centering
    \includegraphics[scale=0.6]{figures/Appbildungen/toolEval.PNG}
    \caption{Bewertung von Process Mining Software (Eigene Darstellung in Anlehnung an: J. Schobel, M. Reichert, 2017, S.244, \cite{Schobel2017})}
    \label{fig:toolEval}
\end{figure}
\newpage
Neben der Bedienfreundlichkeit und dem Umfang der Software sind die ausschlaggebenden Kriterien für die Auswahl der Software diejenigen technischen Daten, die sich direkt auf die praktische Umsetzbarkeit der Testreihe und des Proof of Concept auswirken. Die technischen Eckdaten wurden einer  Abschlussarbeit mit dem Titel "Vergleich und Evaluation von Process Mining Software" der Universität Passau entnommen \cite{compPM} und in Tabelle \ref{comparison2} zusammengefasst.

\begin{table}[!h]
\centering
\resizebox{\textwidth}{!}{%
\begin{tabular}{lccc}
\hline
Kriterium & \multicolumn{1}{c}{\textbf{ProM (V. 6.9)}} & \multicolumn{1}{c}{\textbf{Disco }} & \multicolumn{1}{c}{\textbf{Celonis}} \\
\hline\hline
\textit{Typ des Eingangslogs} & Mxml,xes & Csv,xls,XES,xml, fxl & Csv,Xls,XES \\
\textit{Process Discovery} & Ja & Ja & Ja \\
\textit{Lizenz} & Quelloffen & Kommerziell & Kommerziell \\
\textit{Notation der Ergebnisse} & BPMN, PNML, YAWL, PTML & XML & BPMN \\
\textit{Visualisierung der Prozesse} & Ja & Ja & Ja \\
\textit{lokale Anwendung} & Ja, Client-basiert & Ja, Client-basiert. & Nein, Cloud-basiert \\
\textit{Kommandozeilenanbindung} & Ja & Nein & Nein \\
\hline
\end{tabular}%
}
\caption{Gegenüberstellung von technischen Eigenschaften der populärsten Process Mining Lösungen(Quelle: Erling C., Vergleich und Evaluation von Process Mining Software, 2019, S. 36,37,44,83 \ref{comparison2})}
\label{comparison2}
\end{table}

Ein entscheidender Vorteil von ProM ist, dass einige der integrierten Plugins eine Kommandozeilenschnittstelle bieten. Mit Hilfe dieser kann die Auswertung der Eventlogs von einem Serverseitigen Skript aus gestartet und überwacht werden. Nachteilig an dieser Lösung ist allerdings, dass ProM im Gegensatz zu Disco und Celonis Eventlogs im CSV-Format (Comma Seperated Values) nicht direkt entgegen nimmt. Das heißt es muss eine Konvertierung in das XES Format stattfinden, bevor eine Process Mining Analyse durchgeführt werden kann.

Die quelloffene Lizensierung von ProM, die teils Cloud-basierte Architektur von Celonis und der Mangel einer Kommandozeilenanbindung von Disco sind letztendlich ausschlaggebend für die Entscheidung, ProM als Werkzeug für die Durchführung des Process Mining zu nutzen. 

\section{Wahl des Process Mining Verfahrens}

Es soll ein Process Mining Verfahren identifiziert werden, welches unter den in den Vergleich einbezogenen Minern am besten geeignet ist Modelle derart zu erzeugen, dass sie kurze, häufig auftretende Ereignis und Aktionen korrekt erkennen und abbilden. Ziel ist es, diejenigen Interaktionen mit den Smart Home Geräten des Anwenders herauszufiltern, die innerhalb eines beschränkten Zeitraums wiederholt beobachtet wurden und in einer einfachen Wenn-Dann (If This Then That) Beziehung beschrieben werden können. 

Mining Verfahren, deren Modelle zwar korrekt, aber zu komplex sind um eine für den Anwender leicht interpretierbare Regel zu beschreiben sind für diesen Anwendungsfall ungeeignet. In Anbetracht der im Eingangskapitel erläuterten Gütekriterien werden hier Modelle gefordert, die ein hohes Maß an ‚Simplicity‘ (Einfachheit) mitbringen und weniger Gewicht auf ,Fitness' legen, also kein Modell erstellen, das dazu neigt alle möglichen, selten auftretenden Prozesschritte mit einzuschließen. 

Ein weiterer Faktor, der die Auswahl an Minern einschränkt, ist die praktische Umsetzbarkeit in einer realen Anwendung. Es werden nur solche Miner berücksichtigt, für die zum Zeitpunkt dieser Arbeit ein Plugin in der ProM 6.9 Version angeboten werden und die in der Lage sind, über Kommandozeilenbefehle so aufgerufen zu werden, dass sie ein Ausgabemodell erzeugen. Da das Ziel ist, einen automatisiert ablaufenden Dienst einzurichten, so dass Process Mining ohne zusätzlichen Intervention seitens des Anwenders durchgeführt werden kann.


Es stehen Implementationen des Heuristic, Local, Inductive und des Transitions System Miner zur Verfügung. Um beurteilen zu können, welche der Verfahren sich besonders für die Analyse von Eventlogs eignen, die menschliches Verhalten protokollieren, werden Veröffentlichungen herangezogen, die Process Mining im ADL Kontext eingesetzt haben, insbesondere wenn es sich dabei direkt um den Smart Home Bereich handelt. Die Ergebnisse sind in Tabelle \ref{tab:miningPaper} gelistet. 

Der Local Process Miner ist in der Übersicht nicht aufgeführt, da es bisher keine Veröffentlichungen gibt, die dieses noch junge Verfahren in einem vergleichbaren Kontext untersucht haben. 

\newcolumntype{b}{X}
\newcolumntype{s}{>{\hsize=.7\hsize}c}
\begin{table}[!ht]\small
\centering
\begin{tabularx}{\textwidth}{b ccc}
Titel                                                                                                               & \multicolumn{1}{l}{\textbf{Heuristic}} & \multicolumn{1}{l}{\textbf{Inductive}} & \multicolumn{1}{l}{\textbf{Transition}} \\ \hline\hline
\textit{Discovering process models of activities of daily living from sensors} \cite{adl1}                                      & x                                      &                                        &                                         \\
\textit{Context-based similarity measure on human behavior pattern analysis} \cite{adl2}                                        & x                                      &                                        &                                         \\
\textit{Mining insights from weakly-structured event data} \cite{adl3}                                                          & x                                      & x                                      &                                         \\
\textit{A novel human autonomy assessment system} \cite{adl4}                                                                   &                                        &                                        & x                                       \\
\textit{SAMDY–Technologies to support “Care On Demand”} \cite{adl5}                                                              &                                        &                                        & x                                       \\
\textit{Mining process model descriptions of daily life through event abstraction} \cite{adl6}                                  &                                        & x                                      &                                         \\
\textit{Recompiling learning processes from event logs} \cite{adl7}                                                             &                                        & x                                      &                                         \\
\textit{Event Abstraction for Process Mining Using Supervised Learning Techniques} \cite{adl8}                                  &                                        & x                                      &                                         \\
\textit{Visual process maps: a visualization tool for discovering habits in smart homes} \cite{adl9}                            &                                        & x                                      &                                         \\
\textit{Data Mining for Behavioural Changes and Monitoring Requirements in Residential} Healthcare \cite{adl10}                  & x                                      &                                        &                                         \\
\textit{Prozess-Mining und Prozessbewertung zur Verbesserung klinischer Workflows im Umfeld bilderzeugender Fächer} \cite{adl11} & x                                      &                                        &                                         \\ \hline

\end{tabularx}

\caption{Übersicht über Veröffentlichungen mit dem Thema Process Mining und ADL und das jeweils eingesetzte Process Mining Verfahren}
 \label{tab:miningPaper}
\end{table}
\normalsize
Betrachtet werden hier diejenigen Studien, die eines der Process Mining Algorithmen eingesetzt haben, für das ein Plugin in ProM 6.9 existiert, welches sich über eine Kommandozeilenanbindung ansprechen lässt. Ein weiteres Kriterium war, dass in der Kurzfassung der Veröffentlichung die Absicht beschrieben wurde Daten, die menschliches Verhalten widerspiegeln, über einen Process Mining zu analysieren. Die Suche erfolgte über die Suchmaschine Google Scholar, die gesuchten Begriffe waren eine Kombination des jeweiligen Verfahrens und den Begriffen "ADL", "Smart Home", "Smart Living", "Human Activities" und "Human Behaviour". 

Es lässt sich ein eindeutiger Trend beobachten, der zeigt, dass sowohl der Heuristics als auch der Inductive Miner ein vergleichsweise populäres Mittel sind, um Protokolldateien, die menschliches Verhalten enthalten, auszuwerten, während der Transitions Miner weniger Beachtung findet. Zusätzlich sollen einige Eckdaten zu den Stärken und Schwächen der Verfahren betrachtet werden, wie sie in Tabelle \ref{tab:my-table} aufgeführt sind. Die Informationen wurden den Veröffentlichungen entnommen, die das jeweilige Verfahren vorstellen (Heuristics \cite{heurMining}, Transitions\cite{transMiner}, Inductive \cite{inducIMining}, Local \cite{localMining}). 

\begin{table}[!h]
\centering
\resizebox{\textwidth}{!}{%
\begin{tabular}{lcccc}
 Kategorie & \textbf{Heuristic} & \textbf{Transitions} & \textbf{Inductive} & \textbf{Local}   \\
%Betrachtete Plugin Version & F. Mannhardt, Interactive Data Aware Heur. M. & H.M.W. Verbeek, Mine for a fuzzy Model & S.J.Leemans, Mine with Inductive Visual Miner & N.Tax, Search for Local Process Models \\
\hline\hline
Nebenläufigkeit & + & - & + & + \\
kurze Iterationen (Schleifenlänge ≤ 2) & + & + & - & + \\
Eignung für unvollständige &  &  &  &  \\
Eventlogs & + & + & + & + \\
Eignung für fehlerbehaftete &  &  &  &  \\
Eventlogs & + & + & ? & ? \\
automatisierbar & + & + & + & +\\
\hline
\end{tabular}%
}
\caption{Eigenschaften von Process Mining Verfahren zur Beurteilung ihrer Eignung für den Einsatz im Smart Home}
\label{tab:my-table}
\end{table}

Die Beschaffenheit der Miner und vorrangig der Trend, der in Tabelle \ref{tab:miningPaper} zu erkennen ist, wurde als hinreichende Bewertungsgrundlage angenommen, um den Heuristic Miner und den Inductive Miner zu wählen und sie in einer Testreihe einem Vergleich zu unterziehen. Ziel des Vergleichs ist es zu überprüfen, ob sich eines dieser Mittel eignet, repetitive menschliche Verhaltensmuster bestehend aus nur wenigen Aktionen mit Process Mining zu erkennen.

\newpage
\section{Testreihe}
Um zu untersuchen, inwieweit die Modelle der gewählten Miner sich für den im Rahmen dieser Arbeit geplanten Anwendungsfall eignen, werden sie zunächst mit künstlich erzeugten Ereignisprotokollen evaluiert. Die Testreihe basiert systematisch generierten XES Dateien, die an Eventlogs angelehnt sind, die in einem Smart Home Model des ITOM Instituts entstanden sind. 

\subsection{Aufbau und Durchführung}
Aufgrund der nahezu unbegrenzten Anzahl möglicher Szenarien in einem Smart Home kann nur an einem Ausschnitt an potentiell möglichen zusammengesetzten Ereignisprotokollen getestet werden. Um dennoch eine hohe Aussagekraft der Testreihe zu erhalten, sind die Eventlogs derart gestaltet, dass sie gegenüber dem Miner inkrementell komplexer werden und so Performanzunterschiede zwischen den Algorithmen deutlich werden können. Die Abweichung zwischen gewünschtem Modell und der Ausgabe des jeweiligen Miners soll hier als Grundlage für die Bewertung dienen, um festzustellen inwieweit die untersuchten Miner geeignet sind im Smart Home Kontext zur Erzeugung von Wenn-Dann Regeln eingesetzt zu werden. 

Um eine Vergleichbarkeit mit realen Eventlogs zu ermöglichen, wurden die Logs um Rauscheinträge ergänzt, welche jeweils in ihrem relativen Anteil varrieren. Zusätzlich wurde auch die Zeitspanne der Logs in Kalendertagen varriiert. Eine erfolgreiche Auswertung durch einen Miner hat stattgefunden, wenn das im Ereignisprotokoll eingebettete Muster vollständig im Modell wiedergegeben wird, parallele Abläufe korrekt dargestellt werden und keine überschüssigen (Rausch-) Einträge als Knoten im entstandenen Netz enthalten sind.

Hervorzuheben ist an dieser Stelle, dass die Miner den Inhalt einzelner Einträge nicht deuten, sie unterscheiden die Ressourcen, die den Eintrag erstellen und die zugehörigen Eigenschaften nur Anhand der Zeichenkette im Protokoll. Dementsprechend werden die Namen der IoT Ressourcen und ihrer Aktivitäten hier beispielhaft für gängige vernetzte Geräte gewählt, sie haben keine direkte Auswirkung auf das Ergebnis des Mining Verfahrens und sind beliebig austauschbar. Signifikant für die Auswertung ist zunächst nur die Häufigkeit, mit der eine Ressource einen Eintrag erstellt, und der Zeitpunkt des Eintrags im Protokoll, der zu dieser Ressource gehört.

Die Eventlogs der Testreihe werde mit Hilfe eines im Rahmen dieser Arbeit entstandenen Python Skripts generiert, welches folgende Parameter entgegen nimmt: das eingebettete, zu erkennende Verhaltensmuster, die Menge an konsekutiven Kalendertagen (30 o. 90 Tage) und den Anteil an Rauscheinträgen (15 o. 50), siehe Anhang \ref{lst:generate}. 

Die Testreihe setzt sich aus zwölf verschiedenen Modellen zusammen, die in künstlich erzeugte Eventlogs eingebettet werden. Diese werden inkrementell komplexer und enthalten eine jeweils unterschiedliche Menge an Knoten, die die Elemente einer Regel darstellen. Je Modell (benannt „A“ - „M“) entstehen so 4 unterschiedliche Eventlogs .Für die Durchführung der Testreihe wurden insgesamt 40 Eventlogs generiert, welche 84 individuelle Regeln enthielten; mit minimal einer, maximal vier Regeln pro Eventlog, siehe Tabellen im Anhang \ref{results}. 

\subsection{Auswertung der Versuchsreihe}
Um die Güte der Modelle, die in der Messreihe mit den verschiedenen Minern entstanden sind, zu evaluieren, werden die Ausgaben der Process Mining Plugins mit dem ursprünglich eingebetteten Modell verglichen und der Grad der Abweichung für jeden Datensatz dokumentiert. Um die Performanz der Miner zu quantifizieren, wird die Wiedergabe der eingebetteten Muster im Ausgabemodell über ein Punktesystem bewertet. Als Bewertungsgrundlage der Miner soll der Grad der Genauigkeit, mit der das Modell des jeweiligen Verfahrens die gesuchten Regeln abbildet, herangezogen. 

Entscheidend für den im Rahmen dieser Arbeit betrachteten Anwendungsfall der automatisierten Regelerkennung im Smart Home ist, dass der Mining Algorithmus in der Lage ist, Rauschen aus dem Modell herauszufiltern und korrekt kurze Pfade und sowie ihre Parallelität zu identifizieren. 

Für jedes Ausgabemodell wird die Anzahl der korrekt erkannten Elemente (\textit{True Positive}), sowie die fälschlicherweise im Modell enthaltenen Elemente (\textit{False Postive}) und diejenigen Elemente, die nicht im Modell erscheinen, aber erwartet wurden (\textit{False Negative}), notiert. 
Desweiteren sind Ausgabemodelle nur dann für die Weiterverarbeitung brauchbar, wenn die Elemente, die zu unterschiedlichen Regeln gehören, auch auf unterschiedlichen Pfaden im Modell abgebildet werden. Als weitere Metrik in der Testreihe wird daher die Anzahl der korrekt erkannten Pfade im Ausgangsmodell verwendet (\textit{Parallelism}). Wenn alle Kriterien erfüllt werden, erhält der Miner die volle Punktzahl für das jeweilige Ausgabemodell. Die volle Punktzahl je Ausgabemodell entspricht der Anzahl der eingebetteten Regeln in dem Datensatz.

Die Durchführung der Testreihe hat gezeigt, dass es signifikante Unterschiede in der Fähigkeit der untersuchten Mining Algorithmen gibt, Abläufe mit vielen kurzen Pfaden zu identifizieren und im Ausgabemodell korrekt wiederzugeben. Lediglich für fünf der zwölf Reihen geben beide Miner das gesuchte Resultat fehlerfrei aus (konkret: A,B,D,E,F), siehe Abbildung\ref{tab1} und Abbildung \ref{tab2} im Anhang. In den sieben Reihen, in denen einer der beiden Miner Plugins fehlerhafte Ausgabemodelle produzierte, handelte es sich in fünf Fällen (aus 16 Eventlogs, Reihen: H,J,K,L,M) um das ‚Inductive Visual Miner Plugin‘, wohingegen das ‚Interactive Data-aware Heuristics Miner‘ Plugin lediglich in zwei Modellen, in allen vier zugehörigen Eventlogs (Reihen C,G), unterlag. Die Auswertung der Testreihe hat auch gezeigt, dass beide Miner gut geeignet sind bis zu zwei Regeln in einem Eventlog zu erkennen. Bei regelmäßigem Auftreten der Regel konnten auch kurze Auswertungszeitspannen und ein hohes Maß an Rauscheinträgen das Ausgabemodell nicht verzerren.

Im Unterschied zum Heuristic Miner Plugin gelang es dem Inductive Miner Plugin nicht, mehr als zwei kurze Regeln, die in die Eventlogs der Testreihen H,J,K,L,M eingebettet waren, korrekt in parallel verlaufende Pfaden zu unterteilen. Stattdessen gab der Miner häufig alle gesuchten Elemente auf einem einzigen Pfad des Petri Netzes aus, es gelang ihm nicht, die Parallelität der gesuchten Abläufe zu identifizieren. 
Aus der Perspektive des Inductive Miner Verfahrens betrachtet lies sich also beobachten, dass es dem Inductive Miner Verfahren hier nicht gelang, einen \textit{XOR Cut}, siehe \ref{sec:inducMiner}, an der Stelle zu setzen, an der mehr als zwei kurze parallele Abläufe beginnen. 

\begin{figure}[!ht]
    \centering
    \includegraphics[width=0.9\textwidth,]{figures/Appbildungen/K_inductive_erronousPNG.PNG}
    \caption{Modellierung eines Datensatzes der Testreihe K mit dem Inductive Miner Plugin}
    \label{fig:K_inductive}
\end{figure}
Beispielhaft für diesen Fehlerfall wird das Modell 'K' betrachtet. Eingebettet in die hier eingesetzten Eventlogs lagen drei separate, regelmäßig auftretende Wenn-Dann Abfolgen, bestehend aus jeweils zwei oder drei Elementen. 
Für alle vier Kombinationsmöglichkeiten zur Konfiguration des Eventlogs (30 Tage simulierter Aufnahmezeitraum mit einem Rauschanteil von 15 oder 50 Einträgen und ein Zeitraum von  90 Tagen mit einem Rauschanteil von 15 oder 50) berechnete der Heuristic Miner ein Modell, welches die drei erwarteten Abfolgen enthielt und alle Rauscheinträge herausfiltern konnte, wie in Abbildung \ref{fig:K_heuristic} zu sehen ist.
\begin{figure}[!h]
    \centering
    \includegraphics[width=0.8\textwidth,]{figures/Appbildungen/K_heuristic_correct.PNG}
    \caption{Modellierung eines Datensatzes der Testreihe K mit dem Heuristic Miner Plugin}
    \label{fig:K_heuristic}
\end{figure}
Auch das Inductive Miner Plugin ist in der Lage gewesen das Rauschen herauszufiltern, modellierte aber, bei Eingabe der selben Eventlog Dateien, die das Heuristic Miner Plugin genutzt hat, ein für den Einsatzzweck ungeeignetes Modell. 

Auf Abbildung \ref{fig:K_inductive} ist zu sehen, dass nur eine Regel korrekt in der Wenn-Dann Folge abgebildet wird: auf eine 'Smart Alarm Clock' Aktivität folgt eine 'Coffee Maker' Aktivität. 
Die Komponenten der restlichen eingebetteten Regeln werden zwar erkannt, aber nicht in der korrekten Abfolge, sondern parallel, abgebildet. Diese Form der Abweichung hat sich als charakteristisch für den Inductive Miner herausgestellt. Auch die resultierenden Modelle des Inductive Miner der Reihen H,J,L,M enthielten zwar häufig  die korrekten Elemente, allerdings nicht in der gesuchten Reihenfolge. Es enstanden häufig sogenannte Blumenmodelle, wie sie Eingangs beschrieben wurden, siehe Abschnitt \ref{quality}. Die Modellierung fiel also zu allgemein aus, als dass sie gebraucht werden könnte, um in ein If This Then That Schema überführt zu werden.

Auch der Heuristic Miner generierte nicht für jedes Eventlog das erwartete Modell, beispielsweise gelang es ihm in der Testreihe M nicht, alle Rauscheinträge herauszufiltern. Dennoch wurden alle erwarteten Regeln korrekt dargestellt, siehe Abbildung \ref{fig:M_heuristic}. 
Hierbei ist anzumerken, dass die Testreihe M nur an jedem dritten Tag Einträge zum gesuchten Muster enthielt, das heißt das Verhältnis von Rauscheinträgen zu Regeleinträgen war hier besonders hoch, was höchstwahrscheinlich zu der fehlerhaften Modellierung beigetragen hat.
\begin{figure}[!ht]
    \centering
    \includegraphics[width=0.7\textwidth,]{figures/Appbildungen/M_Heuristic.PNG}
    \caption{Modellierung der Testreihe M aus dem Heuristic Miner Plugin}
    \label{fig:M_heuristic}
\end{figure}

Aus der Auswertung der Testreihe geht hervor, dass aus insgesamt 104 Regeln, die in den 48 Eventlogs eingebettet wurden, der Heuristic Miner 96 korrekt identifiziert hat, der Inductive Miner hingegen nur in 48 Fällen die gesuchten Regeln korrekt abbilden konnte.

\begin{table}[!htbp]
\centering
\resizebox{\textwidth}{!}{%
\begin{tabular}{l|ccccc}
Auswertung                & \multicolumn{1}{l}{\textbf{Testreihen}} & \multicolumn{1}{l}{\textbf{Regeln}} & \multicolumn{1}{l}{\textbf{Fehlerhafte Modelle}} & \multicolumn{1}{l}{\textbf{Identifizierte Regeln}} & \multicolumn{1}{l}{\textbf{Verfehlte Regeln}} \\ \hline
\textit{Inductive M.}  & 48                                      & 104                                 & 14                                               & 48                                                 & 56                                            \\
\textit{Heuristics M.} & 48                                      & 104                                 & 2                                                & 96                                                 & 8                                            
\end{tabular}%
}
\caption{Ergebnisse der Auswertung der Versuchsreihe}
\label{tab:results_short}
\end{table}

Da ein Verfahren gesucht wurde, welches kurze Regeln, also Abläufe bestehend aus zwei bis vier Elementen, erkennt und außerdem in der Lage ist mehrere Regeln, die in einem Eventlog eingebettet sind klar voneinander zu trennen, ging das Heuristic Miner Plugin aus der hier vorgestellten Testreihe als zu bevorzugendes Verfahren hervor.

Neben den Erkenntnissen über die wahrscheinliche Eignung der untersuchten Process Mining Verfahren für den den Einsatz zur automatisierten Regelerkennung, konnte außerdem beobachtet werden, welche Parameter, die den Process Mining Verfahren eigen sind, eine positive oder negative Auswirkung auf die Auswertung haben. 

So gibt es beispielsweise die Möglichkeit, dem Heuristic Miner Plugin "all tasks connected", also "Verbinde alle Abläufe" auf \textit{wahr} zu setzen, was konsistent zu Modellen führt, die von Anwendern und Entwicklern des Process Mining gemeinhin als "Spaghettimodell" bezeichnet werden, wie es hier für einen Datensatz der Testreihe A in Abbildung \ref{fig:A_heuristic_spagh} zu sehen ist.
\begin{figure}[!ht]
    \centering
    \includegraphics[width=0.9\textwidth,]{figures/Appbildungen/A_underfitted.PNG}
    \caption{Modellierung der Testreihe A aus dem Heuristic Miner Plugin}
    \label{fig:A_heuristic_spagh}
\end{figure}

Basierend auf den Erkenntnissen dieser Testreihe wurde das Heuristic Miner Plugin gewählt, um ihn für die Konstruktion des Proof of Concept einzusetzen und desweiteren notiert, dass der Parameter "all tasks connected" auf \textit{false} zu setzen ist.
%Desweiteren deutet die Auswertung der Testreihe darauf hin, dass der Einsatz des Heuristic Miners möglicherweise Prinzipiell gegenüber dem Inductive Miner zu bevorzugen ist, wenn nach einer Modellierung gesucht wird, die dem If This Then That Schema entspricht, also Muster enthält, die jeweils aus mehreren Ketten besttenur wenigen aufeinanderfolgenden Elementen bestehen.





%\Rotatebox{90}{%
%\centering
%    \begin{tabular}{lp{2cm}p{2cm}p{2cm}p{2cm}p{2cm}p{2cm}}
%& Alpha Miner & Heuristic Miner & Heuristic Miner + & Fuzzy %Miner & Transition System Miner & Transition System Miner + %\\
%(garantierte) Ausführbarkeit & - & ? & ? & - & + & + \\
%Nebenläufigkeit & + & + & + & - & + & + \\
%Intervalle & - & - & + & - & + & + \\
%kurze Iterationen (Schleifenlänge ≤ 2) & - & + & + & + & + & %+ \\
%entfernte Abhängigkeiten/ nicht lokales Verhalten & - & + & + %& + & + & + \\
%nicht wahrnehmbare Aktivitäten & - & + & + & - & + & + \\
%Eignung für unvollständige Event Logs & - & + & + & + & + & + %\\
%Eignung für fehlerbehaftete Event Logs & - & + & + & + & - & %- \\
%Frequenz von Traces wird berücksichtigt & - & + & + & ? & - & %+ \\
%parametrisierbar & - & + & + & + & + & + \\
%automatisierbar & + & - & - & - & - & + \\
%tolerant gegenüber unstrukturierten Prozessen & - & ? & ? & + %& - & - \\
%ein Mining-Schritt & + & + & + & + & - & - \\
%zwei Mining-Schritte & - & - & - & - & + & + \\
%Ausgabeformat & & & & & & \\
%Petri-Netz & + & - & - & - & + & + \\
%heuristisches Netz & - & + & + & - & - & - \\
%Fuzzy-Modell & - & - & - & + & - & - \\
%Transitionssystem & - & - & - & - & + & + \\
%\end{tabular} 
%}

%\begin{table}[h!]
%\centering
%\begin{tabular}{ccccccc}
%& Alpha Miner & Heuristic Miner & Heuristic Miner + & Fuzzy Miner & Transition System Miner %& Transition System Miner + \\
%(garantierte) Ausführbarkeit & - & ? & ? & - & + & + \\
%Nebenläufigkeit & + & + & + & - & + & + \\
%Intervalle & - & - & + & - & + & + \\
%kurze Iterationen (Schleifenlänge ≤ 2) & - & + & + & + & + & + \\
%entfernte Abhängigkeiten/ nicht lokales Verhalten & - & + & + & + & + & + \\
%nicht wahrnehmbare Aktivitäten & - & + & + & - & + & + \\
%Eignung für unvollständige Event Logs & - & + & + & + & + & + \\
%Eignung für fehlerbehaftete Event Logs & - & + & + & + & - & - \\
%Frequenz von Traces wird berücksichtigt & - & + & + & ? & - & + \\
%parametrisierbar & - & + & + & + & + & + \\
%automatisierbar & + & - & - & - & - & + \\
%tolerant gegenüber unstrukturierten Prozessen & - & ? & ? & + & - & - \\
%ein Mining-Schritt & + & + & + & + & - & - \\
%zwei Mining-Schritte & - & - & - & - & + & + \\
%Ausgabeformat & & & & & & \\
%Petri-Netz & + & - & - & - & + & + \\
%heuristisches Netz & - & + & + & - & - & - \\
%Fuzzy-Modell & - & - & - & + & - & - \\
%Transitionssystem & - & - & - & - & + & + \\
%\end{tabular} 
%\caption{Eigenschaften der wichtigsten Mining Algorithmen}
%\label{table:1}
%\end{table}