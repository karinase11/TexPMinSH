\chapter{Zusammenfassung und Ausblick}
\section{Zusammenfassung}
\vspace{5mm}
\textsl{Ist die automatisierte Erfassung und lokale Verarbeitung von Abläufen im Smart Home durch Process Mining auf lokalen Servern mit eingeschränkter Rechenleistung möglich?}
\vspace{5mm}

Es konnte gezeigt werden, dass Process Mining prinzipiell dazu geeignet ist, mehrere kurze Verhaltensmuster in emulierten Smart Home Protokolldateien zu entdecken. Unter der Voraussetzung, dass die zu analysierende Protokolldatei regulär autretende Abläufe enthält, und diese sich von den anderen Einträgen in ihrer Auftrittshäufigkeit abheben, sind Process Mining Algorithmen in der Lage, diese Abläufe zu entdecken und zu abstrahieren. Diese Beobachtung legt nahe, dass auch Protokolldateien, die aus realen Smart Living Umgebungen stammen, verwendet werden können, um durch Process Mining kurze, repititive Muster zu extrahieren. 

Der Detektionsprozess war auch mit Ereignisprotokollen, die sich nur über kurze Zeiträume erstrecken, in der Lage die erwarteten Abläufe aufzudecken. Selbst die begrenzten Rechenresourcen eines handelsüblichen Einplatinencomputers reichten aus, um eine automatisierte Regeldetektion durchzuführen, und die erkannte Regel über wenige Verarbeitungsschritte an eine Anwendung auf einem mobilen Endgerät zu übermitteln.

Dem Anwender kann potentiell so zeitintensive Konfigurationsarbeit abgenommen werden, so dass gleichzeitig die Bedienfreundlichkeit im Smart Home erhöht wird. In dem vorgestellten Aufbau wurde Rücksicht auf den, in den Eingangs erwähnten Umfragen geäußerten, Wunsch genommen, private Smart Home Daten des Anwenders zur Auswertung nicht an Dritte zu übergeben. Alle Schritte der Hausautomatisierung und Analyse kommen ohne Interaktion mit einer Cloudinfrastruktur aus und können auf Basis von kleinen Datenmengen arbeiten, die automatisierte Regelerkennung kann isoliert innerhalb eines lokalen Netzwerkes ausgeführt werden.

Der Vergleich von aktuell verbreiteten Process Mining Verfahren und Process Mining Software, hat gezeigt, dass dieses noch junge Forschungsgebiet bisher vorranging in der Industrie produktiv eingesetzt wird und dort Antworten auf ökonomisch relevante Fragen liefert. Dementsprechend ist auch die Mehrheit der Softwareanbieter im Bereich des Process Mining auf Anwendungen im betrieblichen Umfeld fokussiert. 
Für die Forschung und andere, nicht industrielle, Einsatzzwecke ist die Process Mining Produktreihe der 'Process Mining Group' an der Uni Eindhoven die bisher einzige Lösung, welche gleichzeitig den Funktionsumfang und die notwendige Flexibilität bietet wie kommezielle Lösungen. Da ein Großteil der Software, der in der "Process Mining Group" entstanden ist, der Öffentlichkeit lizenzfrei zur Verfügung steht und die Plattform dank Plugins, die von Wissenschaftlern und Entwicklern auf dem Gebiet entwickelt werden, stetig wächst, eignet sie sich ideal um neue Anwendungsgebiete für die Process Mining Technologie zu erproben.
Eine Evaluation mit synthetisieren Daten hat gezeigt, dass das Inductive Miner Verfahren und das Heuristic Miner Verfahren fähig sind, kurze Regeln korrekt zu extrahieren. Im Rahmen der Testreihe hat sich weiterhin ergeben, dass der Heuristic Miner besser geeignet war Modelle derart zu konstruieren, dass sie sich in eine If This Then That Struktur überführen lassen. Dies konnten dann in Form von Regeln in die  Hausautomatisierung integriert werden.

Basierend auf diesen Erkenntnissen wurden die Komponenten des Proof of Concept gewählt, der Gleichzeitig eine Weiterentwicklung eines vorangegangenen Projekts des IT Organisation und Management Institus der Fachhochschule Aachen ist. Es wurde ein Hintergrunddienst umgesetzt, der das Heuristic Miner Verfahren über eine Kommandozeilenanbindung der Software ProM auf gesammelten Smart Home Daten anwendet. Diese werden wiederum von einer Android Anwendung in Regeln umgewandelt, die in eine gegebene openHab Umgebung integriert werden können.
\newpage
\section{Ausblick}
Aus der Untersuchung von Process Mining Verfahren im Smart Living Kontext ergeben sich Fragestellungen in zwei Bereichen: in der Forschung des Process Mining im Bereich des \textit{activites of daily living} (ADL) und in der Entwicklung von Smart Home Anwendungen.

\subsection{Entwicklung von Smart Home Anwendungen}
Sinkende Kosten für die Produktion von Sensoren und elektronischen Bauteilen tragen zur rasant wachsenden Verbreitung von vernetzten Geräten bei, wodurch es als wahrscheinlich gilt, dass auch immer mehr einfache Haushaltsgeräte, wie eta simple Einrichtungsgegenstände, in Zukunft digital vernetzt werden. Die Auswertung von Rohdaten und die Suche nach Mustern in diesen Daten wird durch eine höhere Datendichte erleichtert. Dadurch wächst auch das Potential und die Aussagefähigkeit von Datenanalysen im Smart Living Umfeld.

Die Ergebnisse dieser Arbeit deuten darauf hin, dass der Einsatz von Process Mining prinzipiell geeignet ist, einfache Prozesse aus Eventlogs zu extrahieren, die in einem Smart Home entstehen. Die in dieser Arbeit beschriebene Anwendung profitiert davon, dass die Elemente eines durch Process Mining resultierenden Modelle fast unverändert und unverarbeitet in die Hausautomatisierungssoftware eingegeben werden können, um einen Mehrwert für den Anwender zu generieren. 

Denkbar ist auch die Anreicherung der, vom Smart Home Daten automatisch erzeugten, Datensätze durch weitere Daten des Anwenders, wie Kalendereinträge oder Informationen, die durch andere elektronische Geräte wie sogenannte Smart Watches, gesammelt werden. Auf diese Weise kann die Bandbreite und Präzision der erkannten Abläufe erhöht, und dem Nutzer so noch individueller zugeschnittene Regeln angeboten werden, die sich dynamisch an Änderungen der Tagesplanung des Nutzers anpassen können.

Darüber hinaus könnte der Einsatz des Process Mining im Smart Home insbesondere unterstützend in der Alten- und Krankenpflege eingesetzt werden. Da es durch den demographischen Wandel zu einem akuten Mangel an Pflegekräften kommt, dem aktuelle auch viele weitere technologische Forschungsinitiativen versuchen entgegenzuwirken, wäre eine automatisierte Analyse von Tages- und Pflegeabläufen in Alten- und Pflegeeinrichtungen durch Process Mining möglich. Für dieses Einsatzgebiet gibt es bereits einige wenige Studien, die darauf hinweisen, dass die aus dem Process Mining gewonnen Modelle zur Verbesserung der Abläufe dienen können, auf fehlerhaft durchgeführte Prozesse hinzuweisen, sowie dazu beizutragen Aussagen über den Zustand und die Entwicklung von Patienten zu machen. Auch an dieser Stelle ist davon auszugehen, dass die Auswertung von Daten, die durch Sensoren erfasst werden, welche in der Umgebung der Patienten integriert sind, eine höhere Akzeptanz erfahren, als solche, die durch bildbasierte Methoden wie Überwachungskameras entstehen.

Besonders interessant sind aber auch hier systematische Ansätze, die nicht allein die Rohdaten direkt in Prozessmodelle überführen, sondern kontextuelle Informationen (englisch Context Awareness) in die Analyse miteinbeziehen und über mehrere Eventlogeinträge abstrahieren, so dass das Analyseergebnis insgesamt an Aussagekraft gewinnt.
Beispielsweise kann dann nicht einfach eine erhöhte Herzschlagrate und Temperatur wiedergegeben werden, sondern ein Hinweis darauf ob diese Werte auf eine eine akute Verletzung oder Erkrankung zurückzuführen sind oder darauf, dass sich die Person sportlich betätigt. 

Aus der Relevanz von Kontextinformationen für das Process Mining und dem Bedarf nach Methoden, diese nicht manuell, sondern systematisch und automatisiert aus Rohdaten zu gewinnen, ergeben sich auch wichtige Forschungsfragen für die Datenanalyse durch Process Mining.

Sollen komplexere Abläufe als Regeln verstanden werden, etwa eine automatisierte Klassifizierung von übergeordneten Handlungen wie beispielsweise „Frühstück zubereiten“, indem mehrere Aktionen und Sensorwerte zu einem zusammenhängenden Prozess vereint werden, ist eine aufwendigere Vorverarbeitung der Daten notwendig. Hier ist absehbar, dass es notwendig ist, Daten durch Metainformationen anzureichern um kontextuelle Abhängigkeiten zwischen verschiedenen Eventlogeinträgen herzustellen, etwa durch Clusteringverfahren. 

Diese Ansätze könnten außerdem genutzt werden, um die Modelle aus Local Process Mining Verfahren zu gewichten und so direkt für die in dieser Arbeit vorgestellten Funktion der automatisierten Regelerkennung einzusetzen. Dies würde den Schritt der iterativen Suche nach einem geeigneten Schwellwert in Heuristic Miner Verfahren erübrigen und das Charakteristikum des Local Process Miner, sich auf mehrere kurze Abläufe zu fokussieren, effektiv ausnutzen.

\subsection{Forschungsfragen zu Process Mining}
Nicht nur im Bereich des Smart Living, auch in der Industrie wächst die Qualität der Aussagekraft von Modellen unweigerlich durch die Hinzugabe von Metainformationen zu den gegebenen gesammelten Rohdaten. Gleichzeitig ist die manuelle Eingabe solcher Informationen ein zeit- und kostenintensiv Prozess, der zudem fehleranfällig ist, da er dann auf der menschlichen Beurteilungsgabe basiert.

Die im Rahmen dieser Arbeit eingesetzte Process Mining Analyse ist, wie die allermeisten Process Mining Anwendungen bisher, rein syntaktisch orientiert. Das heißt der Auswertungsprozess von Rohdaten besteht aus Schritten der Datenfilterung und der iterativen Suche nach geeigneten Parametern, bis ein Ergebnis generiert wird, dass eine gegebene Fragestellung hinreichend beantwortet. Process Mining Verfahren differenzieren die Daten, die sie untersuchen aber nicht danach, welchen Inhalt sie widerspiegeln. 

Technologien, die es hingegen ermöglichen, semantische Informationen aus Rohdaten zu extrahieren werden für viele Datenanalyseverfahren gebraucht und entwickelt, entsprechende Lösungen sind jedoch meist auf das jeweilige Verfahren und die gegebene Fragestellung zugeschnitten und sind selten auf andere Anwendungsfälle übertragbar. 

Hieraus ergibt sich der Bedarf an verbesserten Verfahren, die es ermöglichen semantisches Process Mining zu betreiben, welches nicht auf eine einzige Domäne allein zugeschnitten, sondern universell einsetzbar wäre. Die Suche nach kontextuellen Informationen ist aktueller Forschungsgegenstand, dabei werden bereits Ansätze verfolgt, die beispielsweise auf die Berechnung der Levenshtein-Distanz oder dem \textit{  Context Aware Trace Clustering} fußen.

\todo{The Heuristics Miner (Weijters and Ribeiro 2011) and the Fodina algorithm (Vanden
Broucke and De Weerdt 2017), in addition to the directly-follows relation, defines an
eventually-follows relation between activities and allows the process analyst to filter out
infrequent directly-follows and eventually follows relations. Two activities A and B are in
an eventually-follows relation when A is eventually followed by B, before the next appearance
of A or B. The eventually-follows relation, unlike the directly-follows relation, is not
impacted by the presence of chaotic activities. The Heuristic Miner (Weijters and Ribeiro
2011) and Fodina (Vanden Broucke and De Weerdt 2017) both include filtering methods
for the directly-follows and eventually-follows relations that are similar in nature to the filtering
mechanism that is used in the Inductive Miner infrequent (Leemans et al. 2013b).
However, the use of sequential orderings and parallel constructs in the mining approaches
of the Heuristic Miner (DeWeerdt et al. 2011) and Fodina (Vanden Broucke and DeWeerdt
2017) is based on the directly-follows relations only, with the eventually follows relations
being used for the mining of long-term dependencies. Furthermore, in contrast to the Inductive
Miner, the process models discovered with the Heuristic Miner (Weijters and Ribeiro
2011) or Fodina (Vanden Broucke and DeWeerdt 2017) can be unsound, i.e., the can contain
deadlocks}


% https://hal.archives-ouvertes.fr/hal-01298125/
% https://iris.unito.it/handle/2318/1668973


% Existing analysis techniques are purely syntax oriented, i.e., much time is spent on filtering, translating, interpreting, and modifying event logs given a particular question.