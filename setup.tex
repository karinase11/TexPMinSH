%--------
%       package includes
%--------
    % font encoding is set up for pdflatex, for other environments see
    % http://tex.stackexchange.com/questions/44694/fontenc-vs-inputenc
    \usepackage[T1]{fontenc}  % 8-bit fonts, improves handling of hyphenations
    \usepackage[utf8x]{inputenc}
    % provides `old' commands for table of contents. Eases the ability to switch between book and scrbook
    \usepackage{scrhack}
    \usepackage{parskip}

    % ------------------- layout, default -------------------
    % adjust the style of float's captions, separated from text to improve readabilty
    \usepackage[labelfont=bf, labelsep=colon, format=hang, textfont=singlespacing, font=small]{caption}
    
    \usepackage{chngcntr}  % continuous numbering of figures/tables over chapters
    \counterwithout{equation}{chapter}
    \counterwithout{figure}{chapter}
    \counterwithout{table}{chapter}

    % Uncomment the following line if you switch from scrbook to book and comment the setkomafont line
    %\usepackage{titlesec}  % remove "Chapter" from the chapter title
    %\titleformat{\chapter}[hang]{\bfseries\huge}{\thechapter}{2pc}{\huge}
    \setkomafont{chapter}{\normalfont\bfseries\huge}

    \usepackage{setspace}  % Line spacing
    \onehalfspacing
    % \doublespacing  % uncomment for double spacing, e.g. for annotations in correction

    % ------------------- functional,default-------------------
    \usepackage[dvipsnames]{xcolor}  % more colors
    \usepackage{array}  % custom format per column in table - needed on the title page
    \usepackage{graphicx}  % include graphics
    \usepackage{subfig}  % divide figure, e.g. 1(a), 1(b)...
    \usepackage{amsmath}  % |
    \usepackage{amsthm}   % | math, bmatrix etc
    \usepackage{amsfonts} % |
    \usepackage{calc}  % calculate within LaTeX
    \usepackage{algorithm,algpseudocode}
    \usepackage[unicode=true,bookmarks=true,bookmarksnumbered=true,
                bookmarksopen=true,bookmarksopenlevel=1,breaklinks=false,
                pdfborder={0 0 0},backref=false,colorlinks=false]{hyperref}
    \usepackage{hyperref}
    \usepackage[nameinlink,noabbrev]{cleveref}
    \usepackage{wrapfig}
    \usepackage{acronym}
    \usepackage{fancyhdr}
    \fancyhf{}
    \fancyhead[RE,RO]{ \rightmark}
    \fancyfoot[CO,CE] {\thepage}

\usepackage{tabularx}

\renewcommand{\figurename}{Abbildung}
\renewcommand{\contentsname}{Inhaltsverzeichnis}
\renewcommand{\tablename}{Tabelle}
\renewcommand{\bibname}{Literaturverzeichnis}
\renewcommand{\chaptername}{Kapitel}
\renewcommand{\listfigurename}{Abbildungsverzeichnis}

    %==========================================
    % You might not need the following packages, I only included them as they
    % are needed for the example floats
    % ------------------- functional, custom-------------------

    \usepackage{bm}  % bold greek variables (boldmath)
    \usepackage{tikz}
    \usetikzlibrary{positioning}  % use: above left of, etc

    % Improves general appearance of the text
    \usepackage[protrusion=true,expansion=true, kerning]{microtype}

    \usepackage[graphicx]{realboxes}
    \usepackage{adjustbox}

%------- (re)new commands / settings
    % ----------------- referencing ----------------
    \newcommand{\secref}[1]{Section~\ref{#1}}
    \newcommand{\chapref}[1]{Chapter~\ref{#1}}
    \renewcommand{\eqref}[1]{Equation~(\ref{#1})}
    \newcommand{\figref}[1]{Figure~\ref{#1}}
    \newcommand{\tabref}[1]{Table~\ref{#1}}

    % ------------------- colors -------------------
    \definecolor{darkgreen}{rgb}{0.0, 0.5, 0.0}
    % Colors of the Albert Ludwigs University as in
    % https://www.zuv.uni-freiburg.de/service/cd/cd-manual/farbwelt
    \definecolor{UniBlue}{RGB}{0, 74, 153}
    \definecolor{UniRed}{RGB}{193, 0, 42}
    \definecolor{UniGrey}{RGB}{154, 155, 156}


    % ------------------- layout -------------------
    % prevents floating objects from being placed ahead of their section
    \let\mySection\section\renewcommand{\section}{\suppressfloats[t]\mySection}
    \let\mySubSection\subsection\renewcommand{\subsection}{\suppressfloats[t]\mySubSection}


    % ------------------- marker commands -------------------
    % ToDo command
    \newcommand{\todo}[1]{\textbf{\textcolor{red}{(TODO: #1)}}}
    \newcommand{\extend}[1]{\textbf{\textcolor{darkgreen}{(EXTEND: #1)}}}
    % Lighter color to note down quick drafts
    \newcommand{\draft}[1]{\textbf{\textcolor{NavyBlue}{(DRAFT: #1)}}}


    % ------------------- math formatting commands
    % define vectors to be bold instead of using an arrow
    \renewcommand{\vec}[1]{\mathbf{#1}}
    \newcommand{\mat}[1]{\mathbf{#1}}
    % tag equation with name
    \newcommand{\eqname}[1]{\tag*{#1}}


    % ------------------- pdf settings -------------------
    % ADAPT THIS
    \hypersetup{pdftitle={KarinaSchmunkBSC},
                pdfauthor={Karina Schmunk},
                pdfsubject={Bachelorarbeit an der Fachhochschole Aachen},
                pdfkeywords={process mining, smart living,  computer science},
                pdfpagelayout=OneColumn, pdfnewwindow=true, pdfstartview=XYZ, plainpages=true}


    %==========================================
    % You might not need the following commands, I only included them as they
    % are needed for the example floats

    % ------------------- Tikz styles -------------------
    \tikzset{>=latex}  % arrow style


    % ------------------- algorithm ---------------------
    % Command to align comments in algorithm
    \newcommand{\alignedComment}[1]{\Comment{\parbox[t]{.35\linewidth}{#1}}}
    % define a foreach command in algorithms
    \algnewcommand\algorithmicforeach{\textbf{foreach}}
    \algdef{S}[FOR]{ForEach}[1]{\algorithmicforeach\ #1\ \algorithmicdo}

%___________________________________________________________
\usepackage{listings}

\usepackage{color}
\definecolor{gray}{rgb}{0.4,0.4,0.4}
\definecolor{darkblue}{rgb}{0.0,0.0,0.6}
\definecolor{cyan}{rgb}{0.0,0.6,0.6}

\definecolor{pblue}{rgb}{0.13,0.13,1}
\definecolor{pgreen}{rgb}{0,0.5,0}
\definecolor{pred}{rgb}{0.9,0,0}
\definecolor{pgrey}{rgb}{0.46,0.45,0.48}


\lstset{
  basicstyle=\fontsize{9}{11}\selectfont\ttfamily,
  columns=fullflexible,
  showstringspaces=false,
  commentstyle=\color{gray}\upshape
}

\lstdefinelanguage{XML}
{
  morestring=[b]",
  morestring=[s]{>}{<},
  morecomment=[s]{<?}{?>},
  stringstyle=\color{black},
  identifierstyle=\color{darkblue},
  keywordstyle=\color{cyan},
  morekeywords={xmlns,version,type,value,key}% list your attributes here
}

\lstset{language=Java,
  showspaces=false,
  showtabs=false,
  breaklines=true,
  showstringspaces=false,
  breakatwhitespace=true,
  commentstyle=\color{pgreen},
  keywordstyle=\color{pblue},
  stringstyle=\color{pred},
  basicstyle=\ttfamily,
  moredelim=[il][\textcolor{pgrey}],
  moredelim=[is][\textcolor{pgrey}]
}